\chapter{Theoretische Grundlagen}
\section{Oberfl�chenplasmonen}
Ein \emph{Plasma} sei ein Gas aus freien Ladungstr�gern mit Gesamtladung 0 -- so zum Beispiel ein vollst�ndig ionisiertes Gas. Im Rahmen des Drude-Modells der quasifreien Elektronen in einem metallischen Festk�rper kann man die bis auf reibungsartige Kr�fte freien Leitungselektronen als Plasma betrachten. In einem solchen Elektronenplasma als Medium k�nnen sich Ladungstr�gerdichteschwankungen als Wellen fortpflanzen. Man nennt eine solche sich fortpflanzende Plasmawelle \emph{Plasmon}. In einem Volumen aus Plasma gilt f�r eine sich als ebene Welle fortpflanzende Elektronendichteschwankung, dass das erzeugte elektrische Feld stets parallel zum $k_{VP}$-Vektor ist, den man der ebenen Plasmawelle (dem \emph{Volumenplasmon}) zuordnet.
\section{}